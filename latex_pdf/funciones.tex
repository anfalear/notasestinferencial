\chapter{Anexo: Funciones de Python y R para Estadística Inferencial}

Este anexo presenta las funciones más utilizadas en Python y R para la estadística inferencial, acompañadas de ejemplos prácticos. El objetivo es proporcionar una referencia rápida para la implementación computacional de pruebas y procedimientos estadísticos habituales.

\section{1. Pruebas de hipótesis para una media}

\subsection*{Python (\texttt{scipy.stats})}
\begin{itemize}
    \item \texttt{stats.ttest\_1samp(datos, popmean=valor)} \hfill Prueba t para una muestra
    \item \texttt{stats.norm.cdf(z)}, \texttt{stats.norm.ppf(p)} \hfill Funciones de la normal estándar (prueba z)
\end{itemize}

\textbf{Ejemplo:}
\begin{verbatim}
from scipy import stats
import numpy as np
x = np.array([10.2, 9.8, 10.5, 10.1, 9.9])
t_stat, p_val = stats.ttest_1samp(x, popmean=10)
print("t =", t_stat, "p =", p_val)
\end{verbatim}

\subsection*{R}
\begin{itemize}
    \item \texttt{t.test(datos, mu=valor)} \hfill Prueba t para una muestra
    \item \texttt{pnorm(z)}, \texttt{qnorm(p)} \hfill Funciones de la normal estándar (prueba z)
\end{itemize}

\textbf{Ejemplo:}
\begin{verbatim}
x <- c(10.2, 9.8, 10.5, 10.1, 9.9)
t.test(x, mu=10)
\end{verbatim}

\section*{2. Pruebas de hipótesis para dos medias}

\subsection*{Python}
\begin{itemize}
    \item \texttt{stats.ttest\_ind(a, b, equal\_var=True/False)} \hfill Prueba t para dos muestras independientes
    \item \texttt{stats.ttest\_rel(a, b)} \hfill Prueba t para muestras pareadas
\end{itemize}

\textbf{Ejemplo:}
\begin{verbatim}
a = np.random.normal(100, 10, 30)
b = np.random.normal(105, 10, 30)
t_stat, p_val = stats.ttest_ind(a, b, equal_var=False)
\end{verbatim}

\subsection*{R}
\begin{itemize}
    \item \texttt{t.test(a, b, var.equal=TRUE/FALSE)} \hfill Prueba t para dos muestras independientes
    \item \texttt{t.test(a, b, paired=TRUE)} \hfill Prueba t para muestras pareadas
\end{itemize}

\textbf{Ejemplo:}
\begin{verbatim}
a <- rnorm(30, mean=100, sd=10)
b <- rnorm(30, mean=105, sd=10)
t.test(a, b, var.equal=FALSE)
\end{verbatim}

\section*{3. Pruebas de proporciones}

\subsection*{Python (\texttt{statsmodels})}
\begin{itemize}
    \item \texttt{proportions\_ztest([x1, x2], [n1, n2])} \hfill Prueba z para dos proporciones
\end{itemize}

\textbf{Ejemplo:}
\begin{verbatim}
from statsmodels.stats.proportion import proportions_ztest
count = [45, 55]  # éxitos en cada grupo
nobs = [100, 120] # tamaños de muestra
stat, pval = proportions_ztest(count, nobs)
\end{verbatim}

\subsection*{R}
\begin{itemize}
    \item \texttt{prop.test(c(x1, x2), c(n1, n2), correct=FALSE)} \hfill Prueba para dos proporciones
\end{itemize}

\textbf{Ejemplo:}
\begin{verbatim}
prop.test(c(45, 55), c(100, 120), correct=FALSE)
\end{verbatim}

\section*{4. Comparación de varianzas}

\subsection*{Python}
\begin{itemize}
    \item \texttt{stats.levene(a, b)} \hfill Prueba de igualdad de varianzas (Levene)
    \item \texttt{stats.bartlett(a, b)} \hfill Prueba de Bartlett para varianzas
\end{itemize}

\textbf{Ejemplo:}
\begin{verbatim}
from scipy import stats
a = np.random.normal(0, 1, 30)
b = np.random.normal(0, 2, 30)
stat, p = stats.levene(a, b)
\end{verbatim}

\subsection*{R}
\begin{itemize}
    \item \texttt{var.test(a, b)} \hfill Prueba F para dos varianzas
\end{itemize}

\textbf{Ejemplo:}
\begin{verbatim}
a <- rnorm(30, mean=0, sd=1)
b <- rnorm(30, mean=0, sd=2)
var.test(a, b)
\end{verbatim}

\section*{5. Intervalos de confianza}

\subsection*{Python}
\begin{itemize}
    \item \texttt{stats.t.interval(conf, df, loc=media, scale=sem)} \hfill IC para la media (t)
    \item \texttt{stats.norm.interval(conf, loc=media, scale=sem)} \hfill IC para la media (z)
\end{itemize}

\textbf{Ejemplo:}
\begin{verbatim}
media = np.mean(x)
sem = stats.sem(x)
ic = stats.t.interval(0.95, len(x)-1, loc=media, scale=sem)
\end{verbatim}

\subsection*{R}
\begin{itemize}
    \item \texttt{t.test(x)\$conf.int} \hfill IC para la media
\end{itemize}

\textbf{Ejemplo:}
\begin{verbatim}
t.test(x)$conf.int
\end{verbatim}

\section*{6. ANOVA}

\subsection*{Python}
\begin{itemize}
    \item \texttt{stats.f\_oneway(grupo1, grupo2, grupo3, ...)} \hfill ANOVA de un factor
\end{itemize}

\textbf{Ejemplo:}
\begin{verbatim}
g1 = [10, 12, 11, 13]
g2 = [14, 15, 13, 16]
g3 = [11, 10, 12, 13]
f_stat, p_val = stats.f_oneway(g1, g2, g3)
\end{verbatim}

\subsection*{R}
\begin{itemize}
    \item \texttt{aov(y ~ grupo, data=datos)} \hfill ANOVA de un factor
    \item \texttt{summary(anova\_result)} \hfill Resumen del ANOVA
\end{itemize}

\textbf{Ejemplo:}
\begin{verbatim}
grupo <- factor(rep(1:3, each=4))
y <- c(10,12,11,13, 14,15,13,16, 11,10,12,13)
datos <- data.frame(y, grupo)
anova_result <- aov(y ~ grupo, data=datos)
summary(anova_result)
\end{verbatim}

\section*{7. Regresión lineal y correlación}

\subsection*{Python}
\begin{itemize}
    \item \texttt{stats.linregress(x, y)} \hfill Regresión lineal simple y correlación
    \item \texttt{np.corrcoef(x, y)} \hfill Matriz de correlación de Pearson
\end{itemize}

\textbf{Ejemplo:}
\begin{verbatim}
x = np.array([2, 3, 5, 7, 9])
y = np.array([4, 5, 7, 10, 15])
res = stats.linregress(x, y)
print("Pendiente:", res.slope, "Intercepto:", res.intercept, "r:", res.rvalue)
\end{verbatim}

\subsection*{R}
\begin{itemize}
    \item \texttt{lm(y ~ x)} \hfill Regresión lineal simple
    \item \texttt{cor(x, y)} \hfill Correlación de Pearson
\end{itemize}

\textbf{Ejemplo:}
\begin{verbatim}
x <- c(2, 3, 5, 7, 9)
y <- c(4, 5, 7, 10, 15)
modelo <- lm(y ~ x)
summary(modelo)
cor(x, y)
\end{verbatim}

\section*{8. Pruebas de bondad de ajuste y tablas}

\subsection*{Python}
\begin{itemize}
    \item \texttt{stats.chisquare(f\_obs, f\_exp)} \hfill Prueba chi-cuadrado de bondad de ajuste
    \item \texttt{stats.chi2\_contingency(tabla)} \hfill Prueba chi-cuadrado de independencia
\end{itemize}

\textbf{Ejemplo:}
\begin{verbatim}
from scipy.stats import chisquare, chi2_contingency
f_obs = [30, 14, 56]
f_exp = [33, 18, 49]
chi2, p = chisquare(f_obs, f_exp)
\end{verbatim}

\subsection*{R}
\begin{itemize}
    \item \texttt{chisq.test(tabla)} \hfill Prueba chi-cuadrado
\end{itemize}

\textbf{Ejemplo:}
\begin{verbatim}
tabla <- matrix(c(30,14,56,33,18,49), nrow=2, byrow=TRUE)
chisq.test(tabla)
\end{verbatim}

\section*{9. Recursos y documentación}

\begin{itemize}
    \item \texttt{https://docs.scipy.org/doc/scipy/reference/stats.html}
    \item \texttt{https://pandas.pydata.org/}
    \item \texttt{https://www.statsmodels.org/}
    \item \texttt{https://stat.ethz.ch/R-manual/R-devel/library/stats/html/00Index.html}
    \item \texttt{https://cran.r-project.org/web/packages/e1071/}
\end{itemize}
