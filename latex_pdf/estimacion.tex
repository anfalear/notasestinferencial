\chapter{Estimación}

La estadística inferencial constituye uno de los pilares fundamentales de la ciencia estadística, permitiendo **utilizar los resultados derivados de las muestras para obtener conclusiones acerca de las características de una población**. A diferencia de la estadística descriptiva, que se enfoca en organizar y resumir datos observados, la inferencia estadística busca determinar propiedades de una población completa a partir de un subconjunto limitado de datos conocido como muestra.

\begin{definition}[Parámetro y Estadístico]
Un \textbf{parámetro} es un valor numérico que resume una característica de toda la población (por ejemplo, $\mu$, $p$, $\sigma^2$). Un \textbf{estadístico} es una función calculada a partir de los datos de la muestra (por ejemplo, $\overline{X}$, $\hat{p}$, $s^2$).
\end{definition}

Dentro de la inferencia estadística, la **estimación** representa una de las dos áreas principales, siendo la otra las pruebas de hipótesis. La estimación busca proporcionar información sobre los parámetros poblacionales desconocidos mediante el uso de estadísticos muestrales.

\begin{definition}[Estimador y Estimación]
Un \textbf{estimador} es una función de los datos muestrales que se usa para aproximar un parámetro poblacional. El valor numérico que toma el estimador en una muestra concreta se denomina \textbf{estimación puntual}. 

La **estimación por intervalos** proporciona un rango de valores (intervalo de confianza) dentro del cual es probable que se encuentre el parámetro poblacional, junto con un nivel de confianza asociado.
\end{definition}

\begin{remark}
Un buen estimador debe cumplir con ciertas propiedades deseables: ser insesgado (su valor esperado es igual al parámetro), eficiente (mínima varianza), consistente (converge al parámetro al aumentar el tamaño muestral) y robusto (poco sensible a valores atípicos).
\end{remark}

Para comprender adecuadamente la estimación, es crucial dominar el concepto de **distribución muestral**, que representa la distribución de todos los posibles valores de un estadístico que podrían surgir al seleccionar todas las muestras posibles de un tamaño determinado.

\section{Estimación para la Media Poblacional}

La media poblacional ($\mu$) representa el valor promedio de todos los elementos en una población. Es un parámetro constante que no se ve afectado por las observaciones de una muestra particular.

\subsection{Estimación Puntual de la Media}

\begin{definition}[Media Muestral]
El estimador puntual más importante para la media poblacional $\mu$ es la media muestral:
\[
\overline{X} = \frac{1}{n} \sum_{i=1}^n X_i
\]
donde $X_1, X_2, \ldots, X_n$ son las observaciones de la muestra.
\end{definition}

\begin{theorem}[Insesgadez de la Media Muestral]
La media muestral $\overline{X}$ es un estimador insesgado de la media poblacional $\mu$, es decir:
\[
\mathbb{E}[\overline{X}] = \mu
\]
\end{theorem}

\subsection{Estimación por Intervalos para la Media}

\begin{definition}[Intervalo de Confianza]
Un intervalo de confianza para un parámetro poblacional es un rango de valores que, con un nivel de confianza dado $(1-\alpha)$, contiene el valor real del parámetro poblacional.
\end{definition}

La construcción de intervalos de confianza para la media depende de si la desviación estándar poblacional es conocida o desconocida.

\textbf{Caso 1: Desviación estándar poblacional ($\sigma$) conocida}

Cuando se dispone de información histórica abundante o datos previos que permiten conocer $\sigma$, el intervalo de confianza se basa en la distribución normal estándar:

\[
\text{IC}_{1-\alpha}: \left( \overline{X} - z_{\alpha/2} \frac{\sigma}{\sqrt{n}}, \overline{X} + z_{\alpha/2} \frac{\sigma}{\sqrt{n}} \right)
\]

\textbf{Caso 2: Desviación estándar poblacional ($\sigma$) desconocida}

Esta es la situación más común en la práctica. Se utiliza la desviación estándar muestral ($s$) y la distribución t de Student:

\[
\text{IC}_{1-\alpha}: \left( \overline{X} - t_{\alpha/2, n-1} \frac{s}{\sqrt{n}}, \overline{X} + t_{\alpha/2, n-1} \frac{s}{\sqrt{n}} \right)
\]

\begin{remark}
La distribución t de Student es similar a la normal estándar pero tiene "colas" más pesadas, reflejando la incertidumbre adicional de estimar $\sigma$. Los grados de libertad $(n-1)$ son fundamentales para determinar la forma exacta de la distribución.
\end{remark}

\begin{example}[Intervalo de Confianza para la Media]
Una muestra de $n=25$ resistencias eléctricas tiene una media de $\overline{X} = 98$ ohmios y una desviación estándar de $s = 5$ ohmios. Calcule un intervalo de confianza del 95\% para la media poblacional.

\textbf{Solución:}
\begin{enumerate}
\item Nivel de confianza = 95\% $\Rightarrow \alpha = 0.05 \Rightarrow \alpha/2 = 0.025$
\item Grados de libertad = $n-1 = 24$
\item Valor crítico: $t_{0.025,24} \approx 2.064$
\item Error estándar: $SE = \frac{s}{\sqrt{n}} = \frac{5}{\sqrt{25}} = 1$
\item Límites del intervalo:
   \begin{align}
   \text{Límite inferior} &= 98 - 2.064 \times 1 = 95.936\\
   \text{Límite superior} &= 98 + 2.064 \times 1 = 100.064
   \end{align}
\end{enumerate}

Por tanto, el intervalo de confianza del 95\% para la media poblacional es $(95.936, 100.064)$ ohmios.
\end{example}

\subsection{Supuestos y Condiciones de Aplicabilidad}

\begin{theorem}[Teorema del Límite Central]
Si el tamaño de muestra $n$ es suficientemente grande (generalmente $n \geq 30$), la distribución muestral de la media será aproximadamente normal, independientemente de la forma de la distribución poblacional original:
\[
\overline{X} \sim N\left(\mu, \frac{\sigma^2}{n}\right)
\]
\end{theorem}

\begin{remark}
Los supuestos fundamentales para la estimación de la media incluyen:
\begin{itemize}
\item \textbf{Normalidad:} Para muestras pequeñas, se asume que la población sigue una distribución normal.
\item \textbf{Independencia:} Las observaciones deben ser independientes entre sí.
\item \textbf{Robustez:} La distribución t es razonablemente robusta a desviaciones leves de la normalidad, especialmente con muestras grandes.
\end{itemize}
\end{remark}

\section{Estimación para la Proporción Poblacional}

La proporción poblacional ($p$) representa la fracción de elementos en una población que poseen una característica específica de interés. Es fundamental en estudios que involucran datos categóricos o binarios.

\subsection{Estimación Puntual de la Proporción}

\begin{definition}[Proporción Muestral]
El estimador puntual para la proporción poblacional $p$ es la proporción muestral:
\[
\hat{p} = \frac{X}{n}
\]
donde $X$ es el número de elementos en la muestra que poseen la característica de interés y $n$ es el tamaño de la muestra.
\end{definition}

\begin{theorem}[Insesgadez de la Proporción Muestral]
La proporción muestral $\hat{p}$ es un estimador insesgado de la proporción poblacional $p$:
\[
\mathbb{E}[\hat{p}] = p
\]
\end{theorem}

\subsection{Estimación por Intervalos para la Proporción}

Para muestras grandes, cuando se cumplen las condiciones $n\hat{p} \geq 5$ y $n(1-\hat{p}) \geq 5$, el intervalo de confianza se basa en la aproximación normal:

\[
\text{IC}_{1-\alpha}: \left( \hat{p} - z_{\alpha/2} \sqrt{\frac{\hat{p}(1-\hat{p})}{n}}, \hat{p} + z_{\alpha/2} \sqrt{\frac{\hat{p}(1-\hat{p})}{n}} \right)
\]

\begin{example}[Intervalo de Confianza para la Proporción]
En una encuesta realizada a 100 personas, 56 manifestaron preferir un producto específico. Calcule un intervalo de confianza del 95\% para la proporción poblacional.

\textbf{Solución:}
\begin{enumerate}
\item Proporción muestral: $\hat{p} = \frac{56}{100} = 0.56$
\item Nivel de confianza = 95\% $\Rightarrow z_{0.025} = 1.96$
\item Verificación de condiciones: $n\hat{p} = 56 \geq 5$ y $n(1-\hat{p}) = 44 \geq 5$ 
\item Error estándar: $SE = \sqrt{\frac{0.56 \times 0.44}{100}} = \sqrt{0.002464} = 0.0496$
\item Límites del intervalo:
   \begin{align}
   \text{Límite inferior} &= 0.56 - 1.96 \times 0.0496 = 0.463\\
   \text{Límite superior} &= 0.56 + 1.96 \times 0.0496 = 0.657
   \end{align}
\end{enumerate}

El intervalo de confianza del 95\% para la proporción poblacional es $(0.463, 0.657)$.
\end{example}

\begin{remark}
Los contextos típicos donde se aplica la estimación de proporciones incluyen:
\begin{itemize}
\item Encuestas de opinión pública
\item Estudios clínicos y biomédicos
\item Control de calidad en procesos de manufactura
\item Investigación de mercado
\end{itemize}
\end{remark}

\section{Estimación para la Varianza Poblacional}

La varianza poblacional ($\sigma^2$) y su raíz cuadrada, la desviación estándar ($\sigma$), son medidas fundamentales de la dispersión o variabilidad de los datos.

\subsection{Estimación Puntual de la Varianza}

\begin{definition}[Varianza Muestral]
El estimador puntual insesgado para la varianza poblacional $\sigma^2$ es la varianza muestral:
\[
s^2 = \frac{1}{n-1} \sum_{i=1}^n (X_i - \overline{X})^2
\]
\end{definition}

\begin{remark}
La varianza muestral utiliza $(n-1)$ en el denominador en lugar de $n$ para garantizar que sea un estimador insesgado. Esta corrección se conoce como corrección de Bessel.
\end{remark}

\subsection{Estimación por Intervalos para la Varianza}

Cuando la población sigue una distribución normal, el intervalo de confianza para la varianza se basa en la distribución chi-cuadrado ($\chi^2$):

\[
\text{IC}_{1-\alpha}: \left( \frac{(n-1)s^2}{\chi^2_{\alpha/2, n-1}}, \frac{(n-1)s^2}{\chi^2_{1-\alpha/2, n-1}} \right)
\]

\begin{example}[Intervalo de Confianza para la Varianza]
En una muestra de $n=10$ mediciones se obtuvo $s^2 = 2.5$. Calcule un intervalo de confianza del 95\% para la varianza poblacional.

\textbf{Solución:}
\begin{enumerate}
\item Grados de libertad: $n-1 = 9$
\item Valores críticos: $\chi^2_{0.025,9} = 19.02$ y $\chi^2_{0.975,9} = 2.70$
\item Límites del intervalo:
   \begin{align}
   \text{Límite inferior} &= \frac{9 \times 2.5}{19.02} = 1.18\\
   \text{Límite superior} &= \frac{9 \times 2.5}{2.70} = 8.33
   \end{align}
\end{enumerate}

El intervalo de confianza del 95\% para la varianza poblacional es $(1.18, 8.33)$.
\end{example}

\begin{remark}
La estimación de la varianza es crucial en diversas aplicaciones:
\begin{itemize}
\item \textbf{Control de calidad:} Garantizar que un proceso cumpla con especificaciones
\item \textbf{Diseño experimental:} Reducir el error experimental para obtener inferencias más precisas
\item \textbf{Finanzas:} Medir el riesgo asociado a inversiones
\item \textbf{Estudios de capacidad:} Evaluar la capacidad de procesos industriales
\end{itemize}
\end{remark}

\section{Interpretación y Aplicaciones de los Intervalos de Confianza}

\begin{definition}[Nivel de Confianza]
El nivel de confianza $(1-\alpha)$ es la probabilidad de que el intervalo calculado contenga el valor real del parámetro si el procedimiento de muestreo se repitiera muchas veces bajo las mismas condiciones.
\end{definition}

\begin{remark}
Consideraciones importantes sobre los intervalos de confianza:
\begin{itemize}
\item Un intervalo más estrecho indica mayor precisión en la estimación
\item La precisión se puede mejorar aumentando el tamaño de muestra o reduciendo la variabilidad
\item El margen de error es la mitad de la amplitud del intervalo
\item Un resultado estadísticamente significativo no implica necesariamente relevancia práctica
\end{itemize}
\end{remark}

\section{Determinación del Tamaño de Muestra}

Para estimar parámetros poblacionales con un margen de error específico y un nivel de confianza dado, es posible determinar el tamaño de muestra necesario:

\textbf{Para la media (con $\sigma$ conocida):}
\[
n = \left(\frac{z_{\alpha/2} \sigma}{E}\right)^2
\]

\textbf{Para la proporción:}
\[
n = \left(\frac{z_{\alpha/2}}{E}\right)^2 \hat{p}(1-\hat{p})
\]

donde $E$ es el margen de error deseado.

\begin{remark}
La determinación del tamaño de muestra es fundamental en la planificación de estudios, especialmente cuando existen restricciones de presupuesto y tiempo. Para proporciones sin estimación previa, se puede usar $\hat{p} = 0.5$ para obtener el tamaño de muestra más conservador.
\end{remark}

\section{Consideraciones Éticas y Prácticas}

\begin{remark}
Es fundamental que los resultados de estimación se presenten de manera ética e imparcial, incluyendo:
\begin{itemize}
\item Las estimaciones puntuales junto con sus intervalos de confianza
\item El tamaño de muestra utilizado
\item Las suposiciones realizadas
\item Una interpretación clara del significado de los resultados
\item Las limitaciones del estudio
\end{itemize}
\end{remark}

La estimación estadística proporciona las herramientas esenciales para transformar datos muestrales en información confiable que sustente la toma de decisiones en diversas disciplinas, desde la administración y economía hasta la ingeniería y medicina. La correcta aplicación de estas técnicas, junto con una interpretación adecuada de los resultados, constituye la base de la inferencia estadística moderna.
